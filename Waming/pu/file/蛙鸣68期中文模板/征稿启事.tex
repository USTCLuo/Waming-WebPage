\documentclass{warming}

\usepackage{subfig}

\pagestyle{fancy}
\headheight 27pt
\fancyhf{}
\fancyhead[CO]{《蛙鸣》第68期征稿启事}
\fancyhead[CE]{《蛙鸣》第68期征稿启事}
\fancyhead[LO, RE]{}
\fancyhead[LE, RO]{\thepage}

\begin{document}

\setcounter{page}{92}

\chapter{《蛙鸣》第68期征稿启事}

现在,我们正式为《蛙鸣》第68期征稿!

\section*{创刊宗旨}
《蛙鸣》是中国科大数院的学生杂志。1981年6月20日,首期《蛙鸣》由78级数学系的同学们自写、自编、自刻、自印而成。四十余年来,《蛙鸣》一直是一个完全由学生主导,共同探讨、自由交流数学的开放平台,让同学们可以互相交流彼此的思想和发现。所以,我们欢迎各位科大的校友和同学们踊跃投稿!同时,我们也欢迎外校师生投稿,增进交流!

\section*{投稿栏目}
\begin{enumerate}[(1)]
    \item \textbf{蛙鸣记忆}\ 记录属于你的科大故事,叙述你与科大和数学的点点滴滴。
    \item \textbf{初阳}\ 以小论文形式研究一个无需太高深的数学知识就足够理解和欣赏的问题,或对具有一定深度的数学内容进行深入浅出的科普。欢迎大一大二同学来稿!
    \item \textbf{星辰}\ 分享你所喜爱或崇敬的数学工作者的故事、演讲与采访等,或让你印象深刻的讲座、座谈会与研讨会等的记录,或由你所撰写、翻译的数学科普与数学历史等。
    \item \textbf{蛙声一片}\ 以小论文形式记录你对数学更深入的学习或研究,可以撰写关于一个前沿方向或经典问题与相关成果的综述,可以研究一个具有一定深度与背景的数学问题,也可以记录你对数学课程的进阶内容的学习与思考。
    \item \textbf{笔墨诗篇}\ 分享你的原创文学作品,体裁不限。
\end{enumerate}

\section*{投稿步骤}
请将稿件发送到数院学生会官方邮箱:\texttt{mathsu01@ustc.edu.cn}
\begin{enumerate}[(1)]
    \item 邮件标题:蛙鸣\_投稿栏目\_文章标题\_作者。
    \item 邮件正文:作者信息、联系方式(邮箱、QQ或微信等)。
    \item 邮件附件:稿件PDF(请隐去作者信息,以便盲审)。确定录用后请发送稿件tex源码。
\end{enumerate}

\section*{文章要求}

\begin{enumerate}[(1)]
    \item 含有较多公式的文章请使用LaTeX编写,推荐使用《蛙鸣》官网模板。
    \item 数学类文章可以使用中文或英文撰写,要求语言朴实流畅,无明显语病。
    \item 文章摘要:请在正文前简要介绍本篇文章用到的想法、思路、主要内容等概要信息。
    \item 非原创内容请注明参考文献(尽可能详细到章节)。
    \item 其他投稿相关问题可以发送邮件至\texttt{mathsu01@ustc.edu.cn}咨询。
    \item 本期征稿截止日期为2026年4月30日;若因特殊情况需逾期投稿,请联系数院学生会。
\end{enumerate}

更多资讯请关注数院学生会官方QQ(2061453364)与微信公众号(中国科大数学科学学院学生会)了解。

感谢大家对蛙鸣的支持!

\vspace{1cm}

\begin{center}
    \Large\kaishu
    欢迎大家踊跃来稿!
\end{center}

\vspace{1cm}

\captionsetup[subfloat]{labelsep=none,format=plain,labelformat=empty}

\begin{figure}[H]
    \centering
    \subfloat[\small 数院学生会QQ号]{\includegraphics[width=0.2\textwidth]{QQ.jpg}}
    \hspace{1.5cm}
    \subfloat[\small 数院学生会微信公众号]{\includegraphics[width=0.2\textwidth]{wechat.jpg}}
\end{figure}

\end{document}